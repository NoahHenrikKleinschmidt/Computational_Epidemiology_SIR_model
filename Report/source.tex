\documentclass{article}
\usepackage[toc,page]{appendix}
\newcommand{\nocontentsline}[3]{}
\newcommand{\tocless}[2]{\bgroup\let\addcontentsline=\nocontentsline#1{#2}\egroup}

%%%% Bibliography %%%%%
\usepackage[backend=biber, style = numeric, url = false, maxcitenames=1,  citestyle=numeric-comp]{biblatex}
\renewcommand{\citet}[1]{\citeauthor{#1} (\citeyear{#1})}

\addbibresource{Bibliography.bib}
%%%% Bibliography %%%%%

\usepackage[pagestyles]{titlesec}
\usepackage[x11names]{xcolor}
\definecolor{light_yellow}{RGB}{255, 255, 153}
\definecolor{occer}{RGB}{176, 150, 20}
\definecolor{ILIAS_grey}{RGB}{240,240,240}
\definecolor{anthracite}{RGB}{100, 100, 100}
\definecolor{grayish}{RGB}{180, 180, 180}
\definecolor{graylight}{RGB}{220, 220, 220}
\definecolor{unibe}{RGB}{194,1,45}
\usepackage{amsfonts}
\usepackage{amsmath}
\usepackage{textcomp}
\usepackage{enumerate}
\usepackage{tikz}
\usepackage{bibentry}
\usepackage{enumitem}
\usepackage[a4paper]{geometry}
\usepackage{minitoc}
\usepackage{capt-of}
\usepackage{caption}
\setcounter{secnumdepth}{5}
\usepackage{lipsum}
\usepackage{calc}
\usepackage[T1]{fontenc}
\usepackage{etoolbox}
\usepackage[absolute]{textpos}
\usepackage{chngcntr}
\usepackage{ifthen}
\usepackage{titling}
\usepackage{authblk}
\usepackage{booktabs}

\providetoggle{footrule}
\providetoggle{fancytitles}
\providetoggle{headrule}
\let\counterwithout\relax
\let\counterwithin\relax

% --- user editable parameters --- 
\settoggle{footrule}{false} %include footer rule (True) or not (False)
\settoggle{headrule}{false} %include footer rule (True) or not (False)
\settoggle{fancytitles}{false} %use fancy paragraph titles
%\usepackage{showframe}
% --- --- --- --- --- --- --- ---

% --- User editable numbering scheme ---
\counterwithin*{figure}{section}
\counterwithin*{table}{section}
\counterwithin{subsection}{section}
% --- Labelling scheme ---
\renewcommand{\thesection}{\arabic{section}}
\renewcommand{\thesubsection}{\thesection .\arabic{subsection}}
\renewcommand\thesubsubsection{\thesubsection.\arabic{subsubsection}}
\captionsetup[table]{labelfont={color=\titleschemecolor, bf},labelsep=quad, textfont=small }
\captionsetup[figure]{labelfont={color=\titleschemecolor, bf},labelsep= quad, textfont=small }
% --- --- --- --- --- --- --- --- --- ---

\setlength{\TPHorizModule}{30mm}
\setlength{\TPVertModule}{\TPHorizModule}
\textblockorigin{10mm}{10mm} % start everything near the top-left corner
\setlength{\parindent}{0pt}
\dimendef\prevdepth=0
\counterwithin*{figure}{section}
\counterwithin*{table}{section}
\counterwithin{subsection}{section}

\iftoggle{footrule}{
\newcommand{\btm}{1.8cm} 
\newcommand{\fskp}{0.6cm}
\newcommand{\flpa}{0.42}
\newcommand{\flpb}{9.44}
\newcommand{\frpa}{4.1}
\newcommand{\frpb}{\flpb}
}{
\newcommand{\btm}{1.8cm} 
\newcommand{\fskp}{0.22cm}
\newcommand{\flpa}{0.33}
\newcommand{\flpb}{9.44}
\newcommand{\frpa}{4.2}
\newcommand{\frpb}{\flpb}
}

\iftoggle{headrule}{\newcommand{\tp}{25pt}}{\newcommand{\tp}{18pt}}

%abstract formatting
\renewenvironment{abstract}{%
\vspace{-10pt}
\begin{minipage}{0.47\textwidth}
\textcolor{\titleschemecolor}{ABSTRACT}}
{\vspace{5pt}\newline\color{\authorcolor} \titlerule
\end{minipage}}

\newpagestyle{TLS_report_2}{
\newgeometry{a4paper, left = 2cm, right = 2cm, bottom = \btm, top = 0cm, twoside, twocolumn, includehead, includefoot, footskip = \fskp}
\ifthenelse{\equal{\docfont}{helvetica}}{\renewcommand*{\familydefault}{\sfdefault}}{}
\setlength{\columnsep}{0.9cm}
\setlength\headsep{\tp} %used to be 80
\setlength{\headheight}{1.25cm}
\setlength{\voffset}{-1.1in}
\setlength{\topmargin}{0pt}
\newcommand{\hbh}{50}%headbox height
\newcommand{\hbl}{25}%headbox length
\newcommand{\hrl}{0.99\linewidth - \hbl - \hbl}

\renewcommand\Affilfont{\itshape\small}

%Title Formatting
\pretitle{%\vspace{-10pt} 
\begin{minipage}{\linewidth}
\flushleft \bfseries \fontsize{155}{199}\Large\selectfont \textcolor{\titleschemecolor} } 
\posttitle{\vspace{10pt}\end{minipage}
}
\preauthor{\flushleft \large \textcolor{\authorcolor}}
\postauthor{\newline \vspace{10pt}}
\predate{\color{\authorcolor} \titlerule \newline \flushright \vspace{-30pt}}


\iftoggle{fancytitles}{
\titleformat{\section}{\fontsize{14}\large\bfseries\textcolor{\titleschemecolor}}{}{0pt}{\textcolor{\titleschemecolor}{\thesection} \hspace{5pt} \bfseries}%[][\vspace{6pt} {\titlerule[0.9pt]}]


\titleformat{\subsection}{\fontsize{13}\linespread\large\bfseries}{}{0pt}{\textcolor{\titleschemecolor}{\thesubsection} \hspace{5pt} }[\vspace{6pt} {\titlerule[0.9pt]} \vspace{3pt}]
\titlespacing*{\subsection}{0pt}{25pt}{10pt}

\titleformat{\subsubsection}{\fontsize{11}\large\bfseries}{}{0pt}{{\thesubsubsection} \hspace{5pt} \bfseries}[\vspace{-1pt}]
}{
\titleformat{\section}{\fontsize{14}\large\bfseries\textcolor{\titleschemecolor}}{}{0pt}{\textcolor{\titleschemecolor}{\thesection} \hspace{3pt} \bfseries}%[][\vspace{6pt} {\titlerule[0.9pt]}]

\titleformat{\subsection}{\fontsize{13}\linespread\large\bfseries}{}{0pt}{%\textcolor{\titleschemecolor}{\thesubsection} \hspace{5pt} 
	}[]%\vspace{6pt} {\titlerule[0.9pt]} \vspace{3pt}]
\titlespacing*{\subsection}{0pt}{25pt}{10pt}

\titleformat{\subsubsection}{\fontsize{11}\linespread\large\bfseries}{}{0pt}{
	%{\thesubsubsection} \hspace{5pt} \bfseries
	}[\vspace{-1pt}]

}

\sethead[
\makebox(\hbh, \hbl)[b]{
\colorbox{\boxcolor}{\makebox(\hbh, \hbl)[b]{\Large \bfseries  \textcolor{\boxtextcolor} \evenboxcontent \vspace{5pt} }}}
\hspace{3pt}
\iftoggle{headrule}{
\color{\rulecolor}\titlerule
}{}
][][
\begin{textblock}{2}(4.1,0.625)
\begin{flushright}
	\large \bfseries \textcolor{\titleschemecolor}{ \thesection \hspace{4pt} $|$ \hspace{1.5pt} \sectiontitle }
\end{flushright}
\end{textblock}
]
{
\begin{textblock}{5.801}(0.42,0.9255)
\iftoggle{headrule}{ %%%%%%%%%%%
	\color{\rulecolor}\titlerule 
}{}
\end{textblock}
\begin{textblock}{2}(0.42,0.74)
\large \bfseries \textcolor{\titleschemecolor}\doctype
\end{textblock}


}{}{
\begin{textblock}{2}(5.6,0.637)
\makebox(\hbh, \hbl)[b]{
\colorbox{\boxcolor}{\makebox(\hbh, \hbl)[b]{\Large \bfseries  \textcolor{\boxtextcolor} \oddboxcontent \vspace{5pt} }}}
\end{textblock}
}

\setfoot[
\iftoggle{footrule}{\color{\secondaryrulecolor}\titlerule}{}
\begin{textblock}{2}(\flpa,\flpb)
\begin{flushleft}
	\textcolor{\secondaryschemecolor}{ \footleft}
\end{flushleft}
\end{textblock}
][][
\begin{textblock}{2}(\frpa,\frpb)
\begin{flushright}
	\textcolor{\secondaryschemecolor}{ \footright}
\end{flushright}
\end{textblock}
]{
\iftoggle{footrule}{\color{\secondaryrulecolor}\titlerule}{}
\begin{textblock}{2}(\flpa,\flpb)
\begin{flushleft}
	\textcolor{\secondaryschemecolor}{ \footleft}
\end{flushleft}
\end{textblock}
}{}{
\begin{textblock}{2}(\frpa,\frpb)
\begin{flushright}
	\textcolor{\secondaryschemecolor}{ \footright}
\end{flushright}
\end{textblock}
}

}




%customise color scheme
\newcommand{\rulecolor}{grayish}
\newcommand{\secondaryrulecolor}{\rulecolor}

\newcommand{\boxcolor}{grayish}
\newcommand{\boxtextcolor}{white}

\newcommand{\titleschemecolor}{unibe}
\newcommand{\secondaryschemecolor}{\authortitlecolor}

\newcommand{\toccolor}{black}
\newcommand{\authorcolor}{anthracite}

\newcommand{\docfont}{helvetica} %helvetica or times
\newcommand{\headfont}{helvetica} %helvetica or times
%-----------------------

%define what kind of Report is being written in the peripheries
\newcommand{\doctype}{BC I Practical \hspace{6pt} $\boldsymbol{|}$ \hspace{6pt} Report} %odd pages header
\newcommand{\evenboxcontent}{\vspace{-4.5pt} \includegraphics[width = 20pt]{Resources/ub_w}} % content of the box on even pages
\newcommand{\oddboxcontent}{\evenboxcontent} % content of the box on odd pages
\newcommand{\footright}{\thepage} %footer right
\newcommand{\footleft}{\large Communications in Biochemistry} %footer left

%Document Information and Title
\title{Identification of Human Genetic Locus from Unknown Meat Sample by Restriction Fragment Length Polymorphism}
\author[1]{Noah H. Kleinschmidt} %to add affiliation, place the respective sign (number) in [] in front of author, then add \affil with the number and institution.
\affil[1]{Universit\"at Bern}
\date{04.04.2020}

\pagestyle{TLS_report_2}
\begin{document}

\pagenumbering{arabic}
\setcounter{page}{1} 
\maketitle
\thispagestyle{TLS_report_2}

%\begin{abstract}
%	Catalase is an enzyme, charged with protecting cells from oxidative damage by reactive oxygen species hydrogen peroxide. It has been found in all studied aerobic organisms and neutralises reactive $H_2O_2$ into water and molecular oxygen. In this experiment kinetic properties of catalase obtained from potato extract were studied. To that end, oxygenic activity, associated with catalase-mediated $H_2O_2$ decomposition, was monitored. As expected from enzymes that follow Michaelis Menten kinetics, both enzyme and substrate concentrations were found to positively affect catalatic activity. The pH Optimum could be traced around pH 8, which is in accordance with currently published data. Hydroxylamine, an inhibitor of catalase, could be shown to likely affect catalase by a competitive mechanism. \citet{PCRorig}.
%\end{abstract}

%include all Sections that are to be part here...	
%Everything inside the Introduction Section (no Chapters)
\section{Introduction}

This project models the effect of two subgroups within a population. These subgroups are colloquially termed either "normally" susceptible ($S_n$, default average person) or "highly" susceptible ($S_h$, a subgroup that differs from the average by some parameters). $S_h$ are modelled as a linear-transform of $S_n$ and may differ in any combination of infection rate, recovery rate, death rate, and/or relapsation rate (loss of immunity) from $S_n$ by a scalar increase or decrease relative to $S_n$'s respective transition rate.





%\begin{multicols}{3}
\section{Methods and Experiment}

\lipsum

\begin{figure}
	\includegraphics{Resources/ABCD}
	\caption{An example figure of an interaction mapping between arbitrary coloured dots labelled A-D.}
\end{figure}

%\end{multicols}
\section{Modelling}

To apply the formulated model, a \texttt{python} module was developed to numerically solve the outlined system of equations using \texttt{scipy} and \texttt{numpy}. An interactive web-app was developed using \texttt{streamlit} to ease model exploration. 

\large THE APP IS AVAILABLE WHERE??....


\section{Discussion}

\lipsum[9]

\begin{table}
\centering
\caption{HEPES 0.5M, pH7.4  |  500ml}
\begin{tabular}{ll}
\toprule
     Component &        Instruction \\
\midrule
         HEPES &             59.58g \\
        MilliQ &   fill-up to 450ml \\
       NaOH 5M & to pH7.4 (\textasciitilde 17.5ml) \\
        MilliQ &   fill-up to 500ml \\
Sterile Filter &                    \\
\bottomrule
\end{tabular}
\caption*{
This is a buffer recepie table...
}
\end{table}

\subsection{Subsub2}
\lipsum[100]
This is a \footnote{Something of a footnote here...} Text of note.


\section{Supplementary Data}%
%\begin{multicols}{4}
\begin{minipage}{0.3\linewidth}
	%\begin{table}
\centering
\captionof{table}{1.5x LDS  |  2ml}
\begin{tabular}{ll}
\toprule
Component & Instruction \\
\midrule
   4x LDS &       750ul \\
   MilliQ &      1.25ml \\
\bottomrule
\end{tabular}
%\end{table}
\end{minipage}

\begin{minipage}{0.3\linewidth}
%\begin{table}
\centering
\captionof{table}{BSA TBS-Tween  |  50ml}
\begin{tabular}{ll}
\toprule
                 Component &     Instruction \\
\midrule
                       BSA &            2.5g \\
                 TBS-Tween & fill-up to 50ml \\
Before AB add 10\% Na-Azide &          1:1000 \\
\bottomrule
\end{tabular}
%\end{table}
\end{minipage}


\begin{table}
\centering
\caption{HEPES 0.5M, pH7.4  |  500ml}
\begin{tabular}{ll}
\toprule
     Component &        Instruction \\
\midrule
         HEPES &             59.58g \\
        MilliQ &   fill-up to 450ml \\
       NaOH 5M & to pH7.4 (\textasciitilde 17.5ml) \\
        MilliQ &   fill-up to 500ml \\
Sterile Filter &                    \\
\bottomrule
\end{tabular}
\end{table}

%\end{multicols}

\setcounter{section}{\thesection+1}
\printbibliography


\end{document}